\documentclass[12pt]{article}
\usepackage{amsmath}
\usepackage{amssymb}
\usepackage{amsthm}
\usepackage{array}
\usepackage{enumerate}
\usepackage{fancyhdr}
\usepackage{geometry}
\usepackage[colorlinks=true,linkcolor=blue,urlcolor=blue]{hyperref}
\usepackage{listings}
\usepackage{xcolor}
\usepackage{tablefootnote}
\renewcommand{\thefootnote}{%
  \ifcase\value{footnote}%
  \or $\alpha$%
  \or $\beta$%
  \or $\gamma$%
  \or $\delta$%
  \or $\epsilon$%
  \or $\zeta$%
  \or $\eta$%
  \or $\theta$%
  \or $\iota$%
  \else\arabic{footnote}%
  \fi%
}

% Page setup
\geometry{margin=1in}
\setlength{\headheight}{14.49998pt}
\pagestyle{fancy}
\fancyhf{}
\rhead{Samuel Chau - a1799298}
\lhead{MATHS 3026/4026/7026 - Assignment 1}
\cfoot{\thepage}

% Code listing style
\lstset{
    basicstyle=\ttfamily\small,
    backgroundcolor=\color{gray!10},
    frame=single,
    numbers=left,
    numberstyle=\tiny,
}

% Useful commands for crypto
\newcommand{\Z}{\mathbb{Z}}
\newcommand{\N}{\mathbb{N}}
\newcommand{\MOD}[1]{\;(\text{mod}\;#1)}

\begin{document}

%------------------------
% COVER PAGE
%------------------------
\begin{titlepage}
    \centering
    \vspace*{5cm}
    
    \Huge
    \textbf{Assignment 1}
    
    \vspace{1cm}
    
    \Large
    MATHS 3026 / 4026 / 7026 Cryptography
    
    \vfill
    
    \large
    Samuel Chau\\
    a1799298\\
    \vspace{0.5cm}
    \today
\end{titlepage}

%------------------------
% MAIN CONTENT
%------------------------

\noindent\textbf{1.} a) Given $e(x) = 5x + 11 \pmod{26}$ and the word `ALMOND'.

\vspace{0.5cm}

\begin{quote}
i. Encrypt the plaintext ALMOND.

\vspace{0.3cm}

First, we map each letter to its numerical position (A=0, B=1, ..., Z=25):

\begin{center}
\begin{tabular}{|c|c|c|c|c|c|c|}
\hline
Letter & A & L & M & O & N & D \\
\hline
Position & 0 & 11 & 12 & 14 & 13 & 3 \\
\hline
\end{tabular}
\end{center}

\vspace{0.5cm}

Then, using the affine cipher, we can calculate the position for each mapped cipher letter:

\begin{align*}
e(0)  &= 5(0) + 11  &= 11 &= 0 \cdot 26 + 11 &= 11 \pmod{26} &\rightarrow \text{L}\\
e(11) &= 5(11) + 11 &= 66 &= 2 \cdot 26 + 14 &= 14 \pmod{26} &\rightarrow \text{O}\\
e(12) &= 5(12) + 11 &= 71 &= 2 \cdot 26 + 19 &= 19 \pmod{26} &\rightarrow \text{T}\\
e(14) &= 5(14) + 11 &= 81 &= 3 \cdot 26 + 3 &= 3 \pmod{26}  &\rightarrow \text{D}\\
e(13) &= 5(13) + 11 &= 76 &= 2 \cdot 26 + 24 &= 24 \pmod{26} &\rightarrow \text{Y}\\
e(3)  &= 5(3) + 11  &= 26 &= 1 \cdot 26 + 0 &= 0 \pmod{26}  &\rightarrow \text{A}
\end{align*}

\vspace{0.5cm}

The mapped ciphertext:

\begin{center}
\begin{tabular}{|c|c|c|c|c|c|c|}
\hline
Plaintext & A & L & M & O & N & D \\
\hline
Ciphertext & L & O & T & D & Y & A \\
\hline
\end{tabular}
\end{center}

\vspace{0.5cm}

Therefore, the encrypted word is \textbf{LOTDYA}.

\vspace{0.5cm}

ii. Find the decryption algorithm in the form $d(x) = cx + d$ for some $c, d \in \mathbb{Z}_{26}$.

\vspace{0.3cm}

Starting with the encryption function, for clarity we write it as:
$$c = 5p + 11 \pmod{26}$$
where $p$ is the plaintext position and $c$ is the ciphertext position.

To find the decryption function, we solve for $p$:
\begin{align*}
c &= 5p + 11 \pmod{26}\\
c - 11 &= 5p \pmod{26}
\end{align*}

To isolate $p$, we need to divide both sides by 5. However, division is not defined in modular arithmetic, so we instead multiply by the multiplicative inverse of 5. The inverse $5^{-1}$ satisfies $5 \cdot 5^{-1} \equiv 1 \pmod{26}$.

Testing values, we find:
$$5 \cdot 21 = 105 = 4 \cdot 26 + 1 \equiv 1 \pmod{26}$$

Therefore $5^{-1} = 21$. Multiplying both sides of our equation by 21:
\begin{align*}
21(c - 11) &= 21 \cdot 5p \pmod{26}\\
21(c - 11) &= p \pmod{26} \quad \text{(since } 21 \cdot 5 \equiv 1 \pmod{26}\text{)}\\
21c - 231 &= p \pmod{26}
\end{align*}

To reduce $-231 \pmod{26}$:
$$-231 = -9 \cdot 26 + 3 \equiv 3 \pmod{26}$$

Therefore:
$$p = 21c + 3 \pmod{26}$$

In the required form: $\mathbf{d(x) = 21x + 3}$ where $c = 21, d = 3 \in \mathbb{Z}_{26}$.

\vspace{0.3cm}

To verify, we decrypt the ciphertext letter L (position 11) from part i:
\begin{align*}
d(11) &= 21(11) + 3 \pmod{26}\\
&= 231 + 3 \pmod{26}\\
&= 234 \pmod{26}\\
&= 9 \cdot 26 + 0 \pmod{26}\\
&= 0 \pmod{26} \rightarrow \text{A}
\end{align*}

This correctly decrypts L back to A, confirming our decryption algorithm.

\vspace{0.5cm}

iii. Decrypt the ciphertext VSLJF.

\vspace{0.3cm}

First, we map each ciphertext letter to its numerical position:

\begin{center}
\begin{tabular}{|c|c|c|c|c|c|}
\hline
Letter & V & S & L & J & F \\
\hline
Position & 21 & 18 & 11 & 9 & 5 \\
\hline
\end{tabular}
\end{center}

\vspace{0.5cm}

Using the decryption algorithm $d(x) = 21x + 3 \pmod{26}$:

\begin{align*}
d(21) &= 21(21) + 3 &= 441 + 3 &= 444 &= 17 \cdot 26 + 2 &= 2 \pmod{26} &\rightarrow \text{C}\\
d(18) &= 21(18) + 3 &= 378 + 3 &= 381 &= 14 \cdot 26 + 17 &= 17 \pmod{26} &\rightarrow \text{R}\\
d(11) &= 21(11) + 3 &= 231 + 3 &= 234 &= 9 \cdot 26 + 0 &= 0 \pmod{26} &\rightarrow \text{A}\\
d(9)  &= 21(9) + 3  &= 189 + 3 &= 192 &= 7 \cdot 26 + 10 &= 10 \pmod{26} &\rightarrow \text{K}\\
d(5)  &= 21(5) + 3  &= 105 + 3 &= 108 &= 4 \cdot 26 + 4 &= 4 \pmod{26} &\rightarrow \text{E}
\end{align*}

\vspace{0.5cm}

Therefore, the decrypted plaintext is \textbf{CRAKE}.


\end{quote}

\vspace{0.5cm}

\noindent\textbf{b)} Show that you cannot use the affine cipher with the encryption rule $e(x) = 6x + 11 \pmod{26}$ by finding two plaintext letters which encrypt to the same ciphertext letter.

\vspace{0.3cm}

\begin{quote}
To show that this affine cipher doesn't work, we need to find two different plaintext letters that encrypt to the same ciphertext letter.

Let's test the letters A and N:

\vspace{0.3cm}

\textbf{For letter A (position 0):}
\begin{align*}
e(0) &= 6(0) + 11 = 11 \pmod{26} \rightarrow \text{L}
\end{align*}

\textbf{For letter N (position 13):}
\begin{align*}
e(13) &= 6(13) + 11 = 78 + 11 = 89 \equiv 11 \pmod{26} \rightarrow \text{L}
\end{align*}

Both A and N encrypt to the same letter L. This is a collision, which makes decryption impossible.

\vspace{0.5cm}

\end{quote}

%------------------------

\noindent\textbf{2.} a) Encrypt the word CHERRY using the Vigenère cipher.

\vspace{0.5cm}

\begin{quote}
i. Find the ciphertext when the keyword is BROLGA.

\vspace{0.3cm}

\begin{center}
\begin{tabular}{|l|c|c|c|c|c|c|}
\hline
Plaintext Position & 2 & 7 & 4 & 17 & 17 & 24 \\
\hline
Plaintext & C & H & E & R & R & Y \\
\hline
Keyword Position & 1 & 17 & 14 & 11 & 6 & 0 \\
\hline
Keyword & B & R & O & L & G & A \\
\hline
\end{tabular}
\end{center}

\vspace{0.5cm}

Adding the positions modulo 26:
\begin{align*}
2 + 1 &= 3\\
7 + 17 &= 24\\
4 + 14 &= 18\\
17 + 11 &= 28 = 1 \cdot 26 + 2 = 2 \pmod{26}\\
17 + 6 &= 23\\
24 + 0 &= 24
\end{align*}

\vspace{0.5cm}

\begin{center}
\begin{tabular}{|l|c|c|c|c|c|c|}
\hline
Ciphertext Position & 3 & 24 & 18 & 2 & 23 & 24 \\
\hline
Ciphertext & D & Y & S & C & X & Y \\
\hline
\end{tabular}
\end{center}

\vspace{0.5cm}

Therefore, the ciphertext is \textbf{DYSCXY}.

\vspace{0.5cm}

ii. Find the ciphertext when the keyword is MARTIN.

\vspace{0.3cm}

\begin{center}
\begin{tabular}{|l|c|c|c|c|c|c|}
\hline
Plaintext Position & 2 & 7 & 4 & 17 & 17 & 24 \\
\hline
Plaintext & C & H & E & R & R & Y \\
\hline
Keyword Position & 12 & 0 & 17 & 19 & 8 & 13 \\
\hline
Keyword & M & A & R & T & I & N \\
\hline
\end{tabular}
\end{center}

\vspace{0.5cm}

Adding the positions modulo 26:
\begin{align*}
2 + 12 &= 14\\
7 + 0 &= 7\\
4 + 17 &= 21\\
17 + 19 &= 36 = 1 \cdot 26 + 10 = 10 \pmod{26}\\
17 + 8 &= 25\\
24 + 13 &= 37 = 1 \cdot 26 + 11 = 11 \pmod{26}
\end{align*}

\vspace{0.5cm}

\begin{center}
\begin{tabular}{|l|c|c|c|c|c|c|}
\hline
Ciphertext Position & 14 & 7 & 21 & 10 & 25 & 11 \\
\hline
Ciphertext & O & H & V & K & Z & L \\
\hline
\end{tabular}
\end{center}

\vspace{0.5cm}

Therefore, the ciphertext is \textbf{OHVKZL}.

\end{quote}

\vspace{0.5cm}

\noindent\textbf{b)} You receive the ciphertext URWZZZMXZZV.

\vspace{0.5cm}

\begin{quote}
i. What is the keyword if the plaintext is INFORMATION?

\vspace{0.3cm}

To find the keyword, we subtract plaintext positions from ciphertext positions modulo 26.

\begin{center}
\begin{tabular}{|l|c|c|c|c|c|c|c|c|c|c|c|}
\hline
Ciphertext & U & R & W & Z & Z & Z & M & X & Z & Z & V \\
\hline
Ciphertext Position & 20 & 17 & 22 & 25 & 25 & 25 & 12 & 23 & 25 & 25 & 21 \\
\hline
Plaintext & I & N & F & O & R & M & A & T & I & O & N \\
\hline
Plaintext Position & 8 & 13 & 5 & 14 & 17 & 12 & 0 & 19 & 8 & 14 & 13 \\
\hline
\end{tabular}
\end{center}

\vspace{0.3cm}

Finding keyword positions by subtraction:
\begin{align*}
k_1 &= 20 - 8 = 12 \rightarrow \text{M}\\
k_2 &= 17 - 13 = 4 \rightarrow \text{E}\\
k_3 &= 22 - 5 = 17 \rightarrow \text{R}\\
k_4 &= 25 - 14 = 11 \rightarrow \text{L}\\
k_5 &= 25 - 17 = 8 \rightarrow \text{I}\\
k_6 &= 25 - 12 = 13 \rightarrow \text{N}
\end{align*}

The pattern repeats: $k_7 = 12 - 0 = 12$ (M), $k_8 = 23 - 19 = 4$ (E), etc.

Therefore, the keyword is \textbf{MERLIN}.

\vspace{0.5cm}

ii. What is the keyword if the plaintext is APPROPRIATE?

\vspace{0.3cm}

\begin{center}
\begin{tabular}{|l|c|c|c|c|c|c|c|c|c|c|c|}
\hline
Ciphertext & U & R & W & Z & Z & Z & M & X & Z & Z & V \\
\hline
Ciphertext Position & 20 & 17 & 22 & 25 & 25 & 25 & 12 & 23 & 25 & 25 & 21 \\
\hline
Plaintext & A & P & P & R & O & P & R & I & A & T & E \\
\hline
Plaintext Position & 0 & 15 & 15 & 17 & 14 & 15 & 17 & 8 & 0 & 19 & 4 \\
\hline
\end{tabular}
\end{center}

\vspace{0.3cm}

Finding keyword positions by subtraction:
\begin{align*}
k_1 &= 20 - 0 = 20 \rightarrow \text{U}\\
k_2 &= 17 - 15 = 2 \rightarrow \text{C}\\
k_3 &= 22 - 15 = 7 \rightarrow \text{H}\\
k_4 &= 25 - 17 = 8 \rightarrow \text{I}\\
k_5 &= 25 - 14 = 11 \rightarrow \text{L}\\
k_6 &= 25 - 15 = 10 \rightarrow \text{K}\\
k_7 &= 12 - 17 = -5 \equiv 21 \pmod{26} \rightarrow \text{V}\\
k_8 &= 23 - 8 = 15 \rightarrow \text{P}\\
k_9 &= 25 - 0 = 25 \rightarrow \text{Z}\\
k_{10} &= 25 - 19 = 6 \rightarrow \text{G}\\
k_{11} &= 21 - 4 = 17 \rightarrow \text{R}
\end{align*}

The keyword is \textbf{UCHILKVPZGR}.

\vspace{0.5cm}

iii. How many possible plaintexts are there for this ciphertext? What percentage of these are a single English word?

\vspace{0.3cm}

Since the ciphertext URWZZZMXZZV has 11 letters, and each letter can be any of 26 letters in the plaintext (determined by the choice of keyword), there are $26^{11}$ possible plaintexts.

\vspace{0.3cm}

Calculating: $26^{11} = 3,670,344,486,987,776$

\vspace{0.3cm}

Therefore, there are \textbf{3,670,344,486,987,776} possible plaintexts for this ciphertext.

\vspace{0.3cm}

To find how many are English words, we can check a dictionary:

\begin{lstlisting}[language=bash]
grep -E '^.{11}$' /usr/share/dict/words | wc -l
8845
\end{lstlisting}

On Ubuntu, this gives us 8,845 eleven-letter English words out of 3,670,344,486,987,776 total possible plaintexts.

\vspace{0.3cm}

Percentage: $\frac{8,845}{3,670,344,486,987,776} \times 100\% \approx 0.000000000241\%$

\end{quote}

%------------------------

\noindent\textbf{3.} Suppose you have access to two encryption algorithms, called DUV-56 and DUV-69:
\begin{itemize}
\item DUV-56 has a 56 bit key
\item DUV-69 has a 69 bit key
\end{itemize}

Suppose you have sufficient computing power to use an exhaustive key search to find the key of DUV-56 in 24 hours.

\vspace{0.5cm}

\begin{quote}
\textbf{a)} Assuming the two algorithms have a similar computational complexity, approximately how many days would you expect to take to find the key of DUV-69 using an exhaustive key search?

\vspace{0.3cm}

First, find the time to test a single key:

DUV-56 has $2^{56}$ possible keys.

Time to test all DUV-56 keys = 24 hours

Time per key = $\frac{24 \text{ hours}}{2^{56}}$

\vspace{0.3cm}

Now for DUV-69:

DUV-69 has $2^{69}$ possible keys.

Time for DUV-69 = $2^{69} \times \frac{24 \text{ hours}}{2^{56}}$

\vspace{0.3cm}

Simplifying:
\begin{align*}
\text{Time for DUV-69} &= \frac{2^{69} \times 24}{2^{56}} \text{ hours}\\
&= 2^{69-56} \times 24 \text{ hours}\\
&= 2^{13} \times 24 \text{ hours}\\
&= 8192 \times 24 \text{ hours}\\
&= 196,608 \text{ hours}\\
&= \frac{196,608}{24} \text{ days}\\
&= 8,192 \text{ days}
\end{align*}

Therefore, it would take approximately \textbf{8,192 days} to find the key of DUV-69.

\vspace{0.5cm}

\textbf{b)} Suppose that DUV-69 has been designed so that it can be run in two separate stages, so that it is possible to conduct an exhaustive key search for the first 56 bits of a DUV-69 key, followed by a separate exhaustive key search for the last 13 bits. Approximately how many days would you now expect to take to find the key of DUV-69 using an exhaustive key search?

\vspace{0.3cm}

With the two-stage approach:

Stage 1: Test $2^{56}$ possible keys for the first 56 bits = 24 hours (given)

\vspace{0.3cm}

Stage 2: Test $2^{13}$ possible keys for the last 13 bits

From part a, time per key = $\frac{24 \text{ hours}}{2^{56}}$

Time for Stage 2 = $2^{13} \times \frac{24 \text{ hours}}{2^{56}}$

\vspace{0.3cm}

Calculating Stage 2:
\begin{align*}
\text{Time for Stage 2} &= \frac{2^{13} \times 24}{2^{56}} \text{ hours}\\
&= 2^{13-56} \times 24 \text{ hours}\\
&= 2^{-43} \times 24 \text{ hours}\\
&= \frac{24}{2^{43}} \text{ hours}\\
&= \frac{24}{8,796,093,022,208} \text{ hours}\\
&\approx 2.73 \times 10^{-12} \text{ hours (negligibly small)}
\end{align*}

Total time = 24 hours + negligible time $\approx$ 24 hours = 1 day

Therefore, it would take approximately \textbf{1 day} to find the key of DUV-69 using the two-stage approach.

\end{quote}

%------------------------

\noindent\textbf{4.} a) Compute the following, and show your working out.

\vspace{0.5cm}

\begin{quote}
i. Exactly how many decimal digits are needed to write the number $10^{20}$?

\vspace{0.3cm}

$10^{20} = 100,000,000,000,000,000,000$

Counting the digits: 1 followed by 20 zeros = \textbf{21 digits}

\vspace{0.5cm}

ii. Approximately how many binary digits (bits) are needed to write the number $10^{20}$?

\vspace{0.3cm}

Using the approximation $2^{3.3} \approx 10$:

\begin{align*}
10^{20} &\approx (2^{3.3})^{20}\\
&= 2^{3.3 \times 20}\\
&= 2^{66}
\end{align*}

Therefore, approximately \textbf{67 bits} are needed (66 bits can represent up to $2^{66} - 1$, so we need 67 bits to represent $2^{66}$).

\vspace{0.5cm}

iii. Exactly how many binary digits (bits) are needed to write the number $2^{20}$?

\vspace{0.3cm}

$2^{20}$ in binary is 1 followed by 20 zeros: $1\underbrace{00...00}_{20 \text{ zeros}}$

Therefore, exactly \textbf{21 bits} are needed.

\vspace{0.5cm}

iv. Approximately how many decimal digits are needed to write the number $2^{20}$?

\vspace{0.3cm}

Using the approximation $2^{3.3} \approx 10$:

\begin{align*}
2^{20} &= 2^{3.3 \times 6.06...}\\
&\approx 2^{3.3 \times 6}\\
&= (2^{3.3})^6\\
&\approx 10^6\\
&= 1,000,000
\end{align*}

Therefore, approximately \textbf{7 decimal digits} are needed.

\vspace{0.5cm}

v. Summarise your answer using a table.

\vspace{0.3cm}

\begin{center}
\begin{tabular}{|l|c|c|}
\hline
 & $10^{20}$ & $2^{20}$ \\
\hline
Number of decimal digits & 21 & 7 (approx) \\
\hline
Number of binary digits & 67 (approx) & 21 \\
\hline
\end{tabular}
\end{center}

\end{quote}

\vspace{0.5cm}

\noindent\textbf{b)} For each of the following, find the approximate number. Work out the approximate number of decimal and binary digits needed to represent the number.

\vspace{0.3cm}

\begin{quote}

\begin{table}[h]
\begin{center}
\begin{tabular}{|c|l|c|c|}
\hline
Part & Number & Decimal digits & Binary digits \\
\hline
A & 8,000,000,000 & 10 & 33 \\
\hline
B & $10^{11}$\tablefootnote{\href{https://imagine.gsfc.nasa.gov/science/objects/milkyway1.html}{Stars in the Milky Way (NASA)}} & 12 & $\approx 10^{11} = (2^{3.3})^{11} \approx 2^{36.3}$ → 37 \\
\hline
C & $10^{24}$\tablefootnote{\href{https://www.esa.int/Science_Exploration/Space_Science/Herschel/How_many_stars_are_there_in_the_Universe}{Stars in the Universe (ESA)}} & 25 & $\approx 10^{24} = (2^{3.3})^{24} \approx 2^{79.2}$ → 80 \\
\hline
D & $10^{7}$\tablefootnote{\href{https://www.si.edu/spotlight/buginfo/bugnos}{Estimated insect species on Earth (Smithsonian)}} & 8 & $\approx 10^{7} = (2^{3.3})^{7} \approx 2^{23.1}$ → 24 \\
\hline
E & $10^{80}$\tablefootnote{\href{https://en.wikipedia.org/wiki/Observable_universe}{Atoms in the observable universe (Wikipedia)}} & 81 & $\approx 10^{80} = (2^{3.3})^{80} \approx 2^{264}$ → 265 \\
\hline
F & 31,536,000 & 8 & $\approx 10^{7} = (2^{3.3})^{7} \approx 2^{23.1}$ → 24 \\
\hline
G & $2^{64}$ & $2^{64} = (2^{3.3})^{19.4} \approx 10^{19.4}$ → 20 & 65 \\
\hline
H & $2^{128}$ & $2^{128} = (2^{3.3})^{38.8} \approx 10^{39}$ → 39 & 129 \\
\hline
I & $2^{256}$ & $2^{256} = (2^{3.3})^{77.6} \approx 10^{78}$ → 78 & 257 \\
\hline
\end{tabular}
\end{center}
\end{table}

\end{quote}

%------------------------

\noindent\textbf{5.} Suppose we have a symmetric key algorithm with encryption rule $E_k(x)$, and we want to increase the security. An obvious approach is to try a 'double encryption'.

\vspace{0.5cm}

\begin{quote}
\textbf{a)} Suppose we encrypt a plaintext by first using the encryption rule $E_{k_1}$, then using $E_{k_2}$ on the result, that is, use the encryption rule $E(x) = E_{k_2}(E_{k_1}(x))$.

\vspace{0.3cm}

Show that there is a single encryption rule $E_{k_3}(x) = a_3x + b_3 \pmod{26}$ which performs exactly the same encryption.

\vspace{0.3cm}

\textbf{Solution:}

Starting with the two encryption rules:
\begin{align*}
E_{k_1}(x) &= a_1x + b_1 \pmod{26}\\
E_{k_2}(x) &= a_2x + b_2 \pmod{26}
\end{align*}

Now we compose them by applying $E_{k_2}$ to the output of $E_{k_1}$:
\begin{align*}
E(x) &= E_{k_2}(E_{k_1}(x))\\
&= E_{k_2}(a_1x + b_1)\\
&= a_2(a_1x + b_1) + b_2 \pmod{26}\\
&= a_2a_1x + a_2b_1 + b_2 \pmod{26}\\
&= (a_2a_1)x + (a_2b_1 + b_2) \pmod{26}
\end{align*}

This is in the form $a_3x + b_3 \pmod{26}$ where:
\begin{align*}
a_3 &= a_2a_1 \pmod{26}\\
b_3 &= a_2b_1 + b_2 \pmod{26}
\end{align*}

Therefore, the double encryption is equivalent to a single affine encryption with key $k_3 = (a_3, b_3)$.

\vspace{0.5cm}

\textbf{b)} Find the values for $a_3, b_3 \in \mathbb{Z}_{26}$ when $E_{k_1}(x) = 11x + 3 \pmod{26}$ and $E_{k_2}(x) = 5x + 15 \pmod{26}$.

\vspace{0.3cm}

From the given encryption rules: $a_1 = 11$, $b_1 = 3$, $a_2 = 5$, $b_2 = 15$

Using the formulas from part (a):
\begin{align*}
a_3 &= a_2a_1 = 5 \cdot 11 = 55 = 2 \cdot 26 + 3 \equiv 3 \pmod{26}\\
b_3 &= a_2b_1 + b_2 = 5 \cdot 3 + 15 = 30 = 1 \cdot 26 + 4 \equiv 4 \pmod{26}
\end{align*}

Therefore, $E_{k_3}(x) = 3x + 4 \pmod{26}$.

\vspace{0.5cm}

\textbf{c)} Check your solution to part b by:

i. Encrypt the plaintext OK first using $E_{k_1}$ and then encrypt the result using $E_{k_2}$

ii. Encrypt the plaintext OK using $E_{k_3}$

\vspace{0.3cm}

Converting OK to numbers: O = 14, K = 10

\textbf{i.} First apply $E_{k_1}(x) = 11x + 3 \pmod{26}$:
\begin{align*}
E_{k_1}(14) &= 11 \cdot 14 + 3 = 157 = 6 \cdot 26 + 1 \equiv 1 \pmod{26} \rightarrow \text{B}\\
E_{k_1}(10) &= 11 \cdot 10 + 3 = 113 = 4 \cdot 26 + 9 \equiv 9 \pmod{26} \rightarrow \text{J}
\end{align*}

Then apply $E_{k_2}(x) = 5x + 15 \pmod{26}$ to BJ (positions 1, 9):
\begin{align*}
E_{k_2}(1) &= 5 \cdot 1 + 15 = 20 \pmod{26} \rightarrow \text{U}\\
E_{k_2}(9) &= 5 \cdot 9 + 15 = 60 = 2 \cdot 26 + 8 \equiv 8 \pmod{26} \rightarrow \text{I}
\end{align*}

Result: \textbf{UI}

\textbf{ii.} Apply $E_{k_3}(x) = 3x + 4 \pmod{26}$:
\begin{align*}
E_{k_3}(14) &= 3 \cdot 14 + 4 = 46 = 1 \cdot 26 + 20 \equiv 20 \pmod{26} \rightarrow \text{U}\\
E_{k_3}(10) &= 3 \cdot 10 + 4 = 34 = 1 \cdot 26 + 8 \equiv 8 \pmod{26} \rightarrow \text{I}
\end{align*}

Result: \textbf{UI}

Both methods give the same ciphertext, confirming our solution.

\vspace{0.5cm}

\textbf{d)} Suppose an exhaustive key-search attack is applied to a double-encrypted affine ciphertext, is the effective key space increased? (Explain your answer.)

\vspace{0.3cm}

No, the effective key space is not increased.

As shown in part (a), double encryption with two affine ciphers is equivalent to a single affine cipher with parameters $(a_3, b_3)$ where $a_3 = a_2a_1 \pmod{26}$ and $b_3 = a_2b_1 + b_2 \pmod{26}$.

Since the result can always be expressed as a single affine encryption, an attacker only needs to try the same $12 \times 26 = 312$ possible keys to break the cipher, not $312^2$ keys. Therefore, double encryption provides no additional security for the affine cipher.

\end{quote}

%------------------------

\vspace{1cm}
\begin{center}
--- End of Assignment 1 ---
\end{center}

\end{document}